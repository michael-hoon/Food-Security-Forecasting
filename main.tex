\documentclass{article}
\usepackage[utf8]{inputenc}
\usepackage{Mic}
\usepackage[title,titletoc]{appendix}

\geometry
{
 a4paper,
 total={170mm,257mm},
 left=20mm,
 top=20mm,
}

\title{\Bigg01.020 Design Thinking Project III (DDW) \\ Towards Zero Hunger: \\ A Cross-Country Analysis of Factors Affecting Food Security}

\author{Cohort 08 Group 04}

\date{\today}

\begin{document}

\maketitle
\thispagestyle{empty}

\begin{center}
\begin{tabular}{ |c|c|c|c| } 
 \hline
  ID & Name & ID & Name \\
 \hline
  1006617 & Michael Hoon Yong Hau & 1006954 & Joshua John Lee Shi Kai \\
 \hline
  1006651 & Wong Qi Yuan Kenneth & 1007085 & Yaromir Viswanathan \\
 \hline
\end{tabular}
\end{center} 

%\section*{Contributions}

%\begin{itemize}
%    \item Michael Hoon Yong Hau:
%    \item Joshua John Lee Shi Kai:
%    \item Wong Qi Yuan Kenneth:
%    \item Yaromir Viswanathan: 
%\end{itemize}

\tableofcontents

\newpage
\setcounter{page}{1}

\section{Introduction} 
The world faces an unprecedented challenge — to feed a projected population of 9 billion by 2050 while ensuring safe, sustainable, and equitable food systems. This imperative is not only a matter of meeting the demands of a growing population but is deeply rooted in the fundamental right of every individual to access safe, nutritious, and sufficient food. As underscored by the United Nations (UN) Sustainable Development Goal 2: Zero Hunger, the eradication of hunger is not merely an aspiration; it is a moral imperative, a global commitment to securing the well-being of present and future generations. \\

\noindent Amidst this backdrop, our research embarks on a journey to conduct a rigorous cross-country analysis, delving into the intricate web of factors that influence food security on a global scale. By adopting a multidimensional approach, we seek to unravel the complexities of the challenges posed by diverse socio-economic, environmental, cultural, and geopolitical contexts. Our ultimate objective is to contribute to the realization of the UN's vision of a world where no one suffers from hunger — a world where food security is a universal reality. \\ 

\noindent We investigate the extent of climate/economic shocks (\url{https://scholarworks.uvm.edu/cgi/viewcontent.cgi?article=1129&context=calsfac}) against food insecurity among 30 countries in the world, and have found that 

\section{Methodology}

\subsection{Data Collection}
Most of our data is obtained from reputable sources such as the FAO and World Bank Group. Some of the datasets we have found to be useful are 

\subsection{Data Wrangling}
To prepare the raw datasets into cleaned, usable information, we applied several data cleaning techniques, such as normalisation and imputation.

\section{Gauss-Markov Assumptions}
Since we would like to employ a multiple linear regression on our dataset, we need to satisfy the Gauss-Markov theorem to ensure that our estimators are unbiased, efficient, and have minimum variance.

\subsection{Linearity of Data}

\subsection{Normality of Residuals}
We employ the Kolmogorov-Smirnov test for normality of residuals using our data, (Or visualising using a Q-Q plot of the residuals)  

\subsection{Independence of Errors}
To test for independence of the residuals, we employ a Durbin-Watson Test 

\subsection{Heteroskedasticity Analysis}
To test for Homoskedasticity, we can employ a White Test. Visualising the plots of our data, we obtain the \textbf{Skewness} and \textbf{Kurtosis} values: 

\subsection{Multicollinearity Considerations}
To test for multicollinearity, we can use the Variance Inflation Factor (VIF). High VIF values (typically above 10) indicate potential multicollinearity.

\subsection{Endogeneity}

\section{Multiple Linear Regression Model}

Our Multiple Linear Regression Equation is as follows: 

\begin{equation}
    y = ax_1 + bx_2 + cx_3
\end{equation}

\subsection{Hypothesis Testing}
We can conduct an F-test 

\subsection{Statistical Significance}
From the p-values obtained, we can conclude that 

\section{Descriptive Statistics}

\subsection{Adjusted $\mathbb{R}^2$}

\subsection{Bayesian Information Criterion (BIC)}

\section{Model Improvements}

\subsection{Feature Engineering}
Create new features or transform existing ones to better capture relationships with the dependent variable.
\subsubsection{Polynomial Terms}
\subsubsection{Interactive Terms}
\subsubsection{Ridge/Lasso Regularization}

\subsection{Instrumental Variable Regression}

\subsubsection{Omitted Variable Bias and Endogeneity}

\section{Conclusion}

\newpage
\clearpage
\pagenumbering{roman}
\begin{appendices}

\section{Data Visualisation}

\section{Kolmogorov-Smirnov Test}

\section{Q-Q Plot of Error Terms}

\end{appendices}

\newpage
\bibliographystyle{plain}
\bibliography{refs}

\end{document}
