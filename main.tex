\documentclass{article}
\usepackage[utf8]{inputenc}
\usepackage{Mic}
\usepackage[title,titletoc]{appendix}

\geometry
{
 a4paper,
 total={170mm,257mm},
 left=20mm,
 top=20mm,
}

\title{\Bigg01.020 Design Thinking Project III (MU) \\ Towards Zero Hunger: \\ A Cross-Country Analysis of Factors Affecting Food Insecurity}

\author{Cohort 08 Group 04}

\date{\today}

\begin{document}

\maketitle
\thispagestyle{empty}

\begin{center}
\begin{tabular}{ |c|c|c|c| } 
 \hline
  ID & Name & ID & Name \\
 \hline
  1006617 & Michael Hoon Yong Hau & 1006954 & Joshua John Lee Shi Kai \\
 \hline
  1006651 & Wong Qi Yuan Kenneth & 1007085 & Yaromir Viswanathan \\
 \hline
\end{tabular}
\end{center} 

%\section*{Contributions}

%\begin{itemize}
%    \item Michael Hoon Yong Hau:
%    \item Joshua John Lee Shi Kai:
%    \item Wong Qi Yuan Kenneth:
%    \item Yaromir Viswanathan: 
%\end{itemize}

\tableofcontents

\newpage
\setcounter{page}{1}

\section{Introduction} 
The world faces an unprecedented challenge — to feed a projected population of 9 billion by 2050 while ensuring safe, sustainable, and equitable food systems \cite{doi:10.1126/science.1185383}. This imperative is not only a matter of meeting the demands of a growing population but is deeply rooted in the fundamental right of every individual to access safe, nutritious, and sufficient food, as underscored United Nations (UN) Sustainable Development Goal 2: Zero Hunger \cite{unsdg2}. \\

\noindent In light of this, our project aims to conduct a rigorous cross-country analysis, investigating the intricate factors that influence food insecurity on a global scale. Our problem statement is as follows: 

\begin{formal}
    "How might we identify the key factors of influence for food insecurity index across countries, to facilitate informed and targeted governmental policy interventions?"
\end{formal}

\noindent As such, our main target audience for this project is governmental organisations involved in policy-making for their country's food security. This ensures relevant insights to formulate effective policies and interventions addressing food security issues based on specific socioeconomic indicators.

\section{Data Collection and Wrangling}
Most of our data is obtained from reputable sources such as the \Food and Agricultural Organisation of the United Nations (FAO) and the World Bank Group. Some of the datasets we have found to be useful are 

\subsection{Normalisation and Imputation}
To prepare the raw datasets into cleaned, usable information, we applied several data cleaning techniques, such as normalisation and imputation. For normalisation in Excel, we used the built-in \verb|=STANDARDISE()| function, which applies z-score normalisation on each row value. As for data imputation, we applied Mean-Imputation (MI) to replace the missing data values in our dataset. 

\section{Gauss-Markov Assumptions}
Since we would like to employ a multiple linear regression on our dataset, we need to satisfy the Gauss-Markov theorem to ensure that our estimators are unbiased, efficient, and have minimum variance.

\subsection{Linearity of Data}

\subsection{Normality of Residuals}
We employ the Kolmogorov-Smirnov test for normality of residuals using our data, (Or visualising using a Q-Q plot of the residuals)  

\subsection{Independence of Errors}
To test for independence of the residuals, we employ a Durbin-Watson Test 

\subsection{Heteroskedasticity Analysis}
To test for Homoskedasticity, we can employ a White Test. Visualising the plots of our data, we obtain the \textbf{Skewness} and \textbf{Kurtosis} values: 

\subsection{Multicollinearity Considerations}
To test for multicollinearity, we can use the Variance Inflation Factor (VIF). High VIF values (typically above 10) indicate potential multicollinearity.

\subsection{Endogeneity}

\section{Multiple Linear Regression Model}

Our Multiple Linear Regression Equation is as follows: 

\begin{equation}
    \begin{aligned}
        \mathbf{FII} = \alpha & + \beta_1\mathbf{Time} + \beta_2\mathbf{Time} + \beta_3\mathbf{Time} + \beta_4\mathbf{Time} + \beta_5\mathbf{Time} \\ 
        & + \beta_6\mathbf{Time} + \beta_7\mathbf{Time} + \beta_8\mathbf{Time} + \beta_9\mathbf{Time} + \beta_10\mathbf{Time} \\
        & + \beta_{11}\mathbf{Time} + \beta_{12}\mathbf{Time} + \beta_{13}\mathbf{Time} + \beta_{14}\mathbf{Time} + \beta_{15}\mathbf{Time}
    \end{aligned}
\end{equation}

\subsection{OLS Parameter Estimates}
Some descriptive statistics of our table are given in Table \ref{descriptivestats}. Significance levels are indicated with stars below the table. 

\begin{table}[H]
    \centering
        \begin{threeparttable}
            \begin{tabular}{l l l l l}
              \toprule & \textbf{Estimate} & \textbf{Std. Error} & \textbf{t-statistic} & \textbf{p-value} \\ \midrule
            \textbf{$\alpha$} & 54.32765 & 8.65118 & 6.2800 & $\num{4.61e-09}$ $***$ \\ 
            \textbf{$\beta_1$ Time} & 2.66033 & 0.05297 & 50.225 & 0.00000 $***$ \\ 
            \textbf{$\beta_2$ Jan} & 9.18029 & 10.76506 & 0.8530 & 0.39533  \\ 
            \textbf{$\beta_3$ Feb} & -0.23004 & 10.76232 & -0.021 & 0.98298 \\ 
            \textbf{$\beta_4$ Mar} & 32.27630 & 10.75985 & 3.0000 & 0.00324 $***$ \\ 
            \textbf{$\beta_5$ Apr} & 26.53263 & 10.75763 & 2.4660 & 0.01494 $**$ \\ 
            \textbf{$\beta_6$ May} & 28.62230 & 10.75567 & 2.6610 & 0.00876 $***$ \\ 
            \textbf{$\beta_7$ Jun} & 65.79531 & 10.75398 & 6.1180 & $\num{1.02e-08}$ $***$ \\ 
            \textbf{$\beta_8$ Jul} & 102.80165 & 10.75254 & 9.5610 & 0.00000 $***$ \\ 
            \textbf{$\beta_9$ Aug} & 99.89132 & 10.75137 & 9.2910 & 0.00000 $***$ \\ 
            \textbf{$\beta_{10}$ Sep} & 48.56432 & 10.75046 & 4.5170 & $\num{1.38e-05}$ $***$ \\ 
            \textbf{$\beta_{11}$ Oct} & 10.07066 & 10.74980 & 0.9370 & 0.35057 \\ 
            \textbf{$\beta_{12}$ Nov} & -26.33967 & 10.74941 & -2.450 & 0.01559 $**$\\ \bottomrule
            \end{tabular}
            \begin{tablenotes}
                \small
                \item *** $p < 0.01$, ** $p < 0.05$, * $p < 0.1$
            \end{tablenotes}
        \end{threeparttable}
    \caption{\label{descriptivestats}Multiple Linear Regression Model Descriptive Statistics}
\end{table}

\subsubsection{Individual Statistical Significance}
From the \textit{p-values} obtained, we can conclude that 

\subsubsection{Significance Test for Equation}

\subsection{Spurious Relationships}

\section{Descriptive Statistics}

\subsection{Adjusted $\mathbb{R}^2$}

\subsection{Bayesian Information Criterion (BIC)}

\section{Model Improvements}

\subsection{Feature Engineering}
Create new features or transform existing ones to better capture relationships with the dependent variable.
\subsubsection{Polynomial Terms}
\subsubsection{Interactive Terms}
\subsubsection{Ridge/Lasso Regularization}

\subsection{Instrumental Variable Regression}

\subsubsection{Omitted Variable Bias and Endogeneity}

\section{Limitation of Analysis}\label{limitations}
There are certain limitations of this project which we must acknowledge, for a comprehensive analysis of the suggested outcomes. 

\section{Conclusion}

\newpage
\clearpage
\pagenumbering{roman}
\begin{appendices}

\section{Data Visualisation}

\section{Kolmogorov-Smirnov Test}

\section{Q-Q Plot of Error Terms}

\end{appendices}

\newpage
\bibliographystyle{plain}
\bibliography{refs}

\end{document}
